%!TEX root = ../dissertation.tex

\chapter{Introduction}


\section{Motivation}

Mobile robots that assist humans in their domestic tasks can increase the
quality of life of everyone in this world. While the ultimate goal of
\textit{AI} is to produce a totally autonomous agent that makes rational
decisions on its environment and learns from it, making these
decisions is generally a hard thing to do. Regularly, there is a broad source of
uncertainty coming from lousy sensors and unexpected action effects which makes
the problem of decision making very difficult. Not only this can generally
happen, but oftentimes these robot have several limitations on some of the
actions they must perform. Several ways to overcome these limitations exist,
ranging from better sensors and actuators, to agents that seek help from
other agents to reach their goals. However, if the other agent is a robot, it
can also have the same limitations, making its help worthless. Consequently, the
help it gets from the other agents depend on their own nature: if they provide
some observation data, its reliability must be taken into account; the ability
to make certain action by other agent is also important, as the execution of
some action may be severely compromised. Most of these problems can be largely
minimized by asking help to a nearby available human agent, as the human is
capable of executing most of the household tasks when the robot is unsure if
he will be able to perform it. Finally, by formalizing and solving this problem,
the approach taken can be generalized to solve other kinds of problems where
there is uncertainty in the world model and rewards for executing some action.

\section{Problem Description}

Considering the multiple tasks the mobile robot must complete in the domestic
setting, the existing uncertainty in both outcomes of agent's actions and
observations he receives from the environment, which can be modeled as a
\textit{Partially Observable Markov Decision Process}, the
agent should always perform the actions that lead him to the best outcome
possible. In order to achieve this result, the robot agent should plan in
advance, taking this uncertainty and possible outcomes into account. However,
this is not a simple problem to solve, as the \textit{physics} of the this
domain must be described in some formal way that is expressive enough to detail
everything that happens within it. Another issue is related to the procedure
that should be used to solve the planning problem, as the computational
resources needed to get an optimal plan usually exceeds the duration of the task
itself. Tipically, this is where the challenge is found:

\textit{"What procedure should the planner have for finding a
quasi-optimal plan which takes as less time as possible to compute and uses few
computational resources?}

Hence, finding a planner that obeys the conditions established before is not an
easy task, and many times one is sacrificed in favor of the other to achieve
better plans, which could be not advisable in some circumstances.


\section{Objectives}
There is a considerable number of objectives that are expected to be
accomplished with this thesis, but in short:
\begin{itemize}
    \item Study the state of the art planning languages for domains coupled with
    uncertainty.
    \item Analyze and Benchmark state of the art planners with varied domains.
    \item Develop a new planner that uses a different strategy to find the best
    decision set.
    \item Benchmark the system made against other teams from the RoboCup@home
    competition.
\end{itemize}
