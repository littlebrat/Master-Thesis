%!TEX root = ../dissertation.tex

\chapter{RDDL Example}

\definecolor{Brown}{cmyk}{0,0.81,1,0.60}
\definecolor{OliveGreen}{cmyk}{0.64,0,0.95,0.40}
\definecolor{CadetBlue}{cmyk}{0.82,0.77,0.13,.4}
\definecolor{lightlightgray}{gray}{0.9}

\lstset{
%language=Java,                             % Code langugage
basicstyle=\ttfamily \small,                   % Code font, Examples: \footnotesize, \ttfamily
keywordstyle=\color{CadetBlue},        % Keywords font ('*' = uppercase)
commentstyle=\color{OliveGreen},              % Comments font
numbers=left,                           % Line nums position
%numberstyle=\tiny,                      % Line-numbers fonts
%stepnumber=1,                           % Step between two line-numbers
numbersep=5pt,                          % How far are line-numbers from code
%backgroundcolor=\color{lightlightgray}, % Choose background color
%frame=none,                             % A frame around the code
tabsize=4,                              % Default tab size
%captionpos=b,                           % Caption-position = bottom
%breaklines=true,                        % Automatic line breaking?
%breakatwhitespace=false,                % Automatic breaks only at whitespace?
showspaces=false,                       % Dont make spaces visible
showtabs=false,                         % Dont make tabls visible
showstringspaces=false,
%columns=flexible,                       % Column format
morecomment=[l]{//},
morekeywords={domain, types, bool, int, real, default, state, action, interm, observ, fluent, requirements, reward, deterministic, continuous, multivalued, intermediate, nodes, constrained, partially, observed, concurrent, integer, valued, pvariables, cpfs, cdfs, if, then, else, Bernoulli, Normal, KronDelta, Discrete, reward, instance, init, state, true, false, max, nondef, actions, horizon, discount, objects, non, fluents, constraints}
}

\begin{lstlisting}[label = code:rddl]
////////////////////////////////////////////////////////////////////////
// 				  Home Environment Domain with possible               //
//                   Symbiotic Relationship example                   //
////////////////////////////////////////////////////////////////////////
domain symbiotic_home {

	// The requirements for this domain should be expressed next:
	requirements = { reward-deterministic };

	// Here, the state and actions fluents are enumerated. This domain
	// is not exactly the same that was discussed before in the logical
	// programming language examples.

	pvariables {
		have_object : { state-fluent,  bool, default = false };
		reachable : { state-fluent,  bool, default = false };
		grab : { action-fluent, bool, default = false };
		ask_grab : { action-fluent, bool, default = false };
	};

	// Conditional probability functions for the next states.
	cpfs {
		have_object = if (~have_object ^ reachable ^ grab) then Bernoulli(0.95)
					  else if (~have_object ^ ~reachable ^ ask_grab)
					  then Bernoulli(0.8);
	};

	// The reward for each state will be given by the following expression.
	reward = have_object * 10 - grab * 5 - ask_grab * 10;
	// Since it handles with boolean values only, they will modified to 1 or
	// 0 depending if they are true of false.
}

// Here it is defined a problem instance of the domain above
instance problem_symbiotic_home {

	domain = home_symbiotic;
	init-state {
		// As it was stated before in the domain file, the default values
		// for the states is false so the following lines can be ommited.
		// have_object = false;
		// reachable = false;
	};

	// The agent can not perform more than one action at a time.
	max-nondef-actions = 1;

	// The agent should plan with a finite horizon accordingly
	// to the following time steps.
	horizon  = 6;

	// The discount factor for rewards in each time time step is
	// described next.
	discount = 0.85;
}
\end{lstlisting}
