%TEX root = ../dissertation.tex

\chapter{Research Plan Schedule}

The structure of activities to be developed along the following semester and
beyond is divided in the following sequential blocks:
\begin{itemize}
    \item Analyze the research done in the field of First Order Markov Decision
    Processes that seem very relevant to the present problem that is being
    discussed.\\
    \textbf{Proposed conclusion time: } Mid-February.
    \item Review, test, benchmark and analyze the performance of the state of
    the art planners that competed in the previous \textit{IPC} from
    \textit{ICAPS}. \\
    \textbf{Proposed conclusion time: } Late-March.
    \item Study of how can we adapt DT-Problog to be a general solver for
    planning problems and if it can outperform the state of the art planners.\\
    \textbf{Proposed conclusion time: } Mid-April.
    \item Model the RDDL planning language in a compatible way to our planner so
    that it can solve problems of some specified domain.\\
    \textbf{Proposed conclusion time: } Mid-May.
    \item Implementation and testing of the planner that was designed in
    the set of domains and respective problems from the previous \textit{IPC}.
    This is followed by a discussion of the method implemented.\\
    \textbf{Proposed conclusion time: } Mid-June.
    \item Study of how can we model the domain of Symbiotic Autonomy with
    \textit{MORDOMO} robot, by looking into the research already done by the
    \textit{CoBot} group from Carnegie Mellon University.\\
    \textbf{Proposed conclusion time: } Late-June.
    \item Integrate the solution developed before into the ROS platform and test
    it with \textit{MORDOMO} robot. Analyze the results obtained when the
    planner runs along with other blocks of the robot, as it may turn out to be
    computationally expensive to the system as a whole.\\
    \textbf{Proposed conclusion time: } Late-June.
    \item Participate in RoboCup@Home with MORDOMO where several capabilities of
    the planner will be put into test in different domains and will have to
    compete against other teams.\\
    \textbf{Proposed conclusion time: } Late-June to Early-July.
\end{itemize}

It is important to note that these are only time estimates of the main tasks, it
may (and it will) occur unexpected delays in any of them, so it was
estabilished an intern time buffer to cope with some of these problems, but even
that may be in vain as they could prove to be more complex that looked at first
sight. Another relevant remark is that it was decided not to model this schedule
as a GANTT chart as its structure is very rigid and if some delay would happen,
the chart would be almost worthless, that is why it is described as a list of
sequential-timed goals which can be easily modified to deal with changes.
