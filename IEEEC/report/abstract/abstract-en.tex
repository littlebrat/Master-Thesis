%!TEX root = ../dissertation.tex

\begin{otherlanguage}{english}
\begin{abstract}
% Set the page style to show the page number
\thispagestyle{plain}
\abstractEnglishPageNumber
A domestic robot that is expected to do tasks in collaboration with humans can
seek help if he is aware that he cannot execute some action or if it is
more advantageous to each one of them that the human executes the action
for him. This robot should make a plan that incorporates the uncertainty about
the world in his model, so that he can achieve the highest utility possible.

But planning in these kind situations where the outcome of an action
is uncertain and the information about the world is incomplete or even incorrect
has proven to be a difficult set of problems in planning and
multiple approaches have been used to solve it. Unfortunately, most
state-of-the-art algorithms take significant computational resources to solve
them and in most of the cases, the solution reached is not optimal.

The research done focus on planning while interacting on uncertain environments
so that the robot can always take the decision that provides the highest
expected utility, when possible.

It is discussed the use of the Decision-Theoretic-Problog (DT-Problog)
framework against state-of-the-art Monte-Carlo methods, to solve the type of
problems mentioned above.

Finally, it is also conducted a detailed study on symbiotic-autonomous service
robots, ranging from the interaction with the human counterpart to the
information that these robots must provide, so that the human agent can decrease
the uncertainty of the robot state in his world model or even do certain action
for them if they are not able to accomplish it, because of its phisical
limitations.

% Keywords
\begin{flushleft}

\keywords{Planning Under Uncertainty, DT-Problog, Symbiotic Autonomy}

\end{flushleft}

\end{abstract}
\end{otherlanguage}
