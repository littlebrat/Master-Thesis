%!TEX root = ../dissertation.tex

\begin{otherlanguage}{portuguese}
\begin{abstract}
\abstractPortuguesePageNumber


A existência de robôs que interagem com pessoas em ambientes domésticos,
requere um novo grau de autonomia e raciocínio. Tipicamente, este tipo de 
ambientes desafia as capacidades do robot (p.e. ter de apanhar um objecto que 
se encontra inacessível). Da mesma forma, algumas acções realizadas pelo agente 
robótico têm uma maior probabilidade de successo devido ao seu conhecimento do ambiente.
A Cooperação entre humanos e robôs pode ajudar a ultrapassar estes problemas, resultando
num maior número de tarefas possíveis para o robot. Este é um exemplo de
Autonomia Simbiótica.

Neste cenário, existe um momento em que o robot tem a opção de realizar uma ação
por ele próprio ou pedir ajuda a um agente humano caso esta escolha os beneficie
de igual forma - a curto ou longo prazo.

Para lidar com este tipo de problemas é proposta a utilização de um sistema
de planeamento baseado no planeador \textit{HYPE}, que em cada
passo adquire observações vindas do ambiente, gerando um problema de decisão
markoviano com variáveis de estado determinadas por estas mesmas observações
a partir do domínio descrito e, decidindo a ação que deve tomar de forma a
maximizar a sua \textit{performance} neste ambiente.
Finalmente, este sistema é testado num ambiente de simulação com
observações geradas previamente, bem como numa situação real com agentes humanos
e um robot.


% Keywords
\begin{flushleft}

\palavrasChave{Autonomia Simbiótica, Planeamento Sob Incerteza, Programação
Lógica Probabilística}

\end{flushleft}

\end{abstract}
\end{otherlanguage}
